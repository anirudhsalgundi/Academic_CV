\vspace{-0.7em}
\begin{section}{Research Experience}
    \begin{subsection}{1. Research Assistant}{Research Supervisor - Prof. Varun Bhalerao \href{https://www.star-iitb.in/home}{(STAR Lab)}}{\textit{Jan 2023 -- Present}}{\textit{Indian Institute of Technology, Bombay}}
        \vspace{0.5em}
            \item \textit{\textbf{AstroSat}}: Studying a sample of 15 thermonuclear X-ray Bursts from two transient Low Mass X-ray Binary sources 4U\:1728--34 \&  4U\:1735--44 using \textit{AstroSat} data.
            \item Developed pipelines for basic data reduction, time resolved burst spectral analysis and timing analysis for exploring accretion phenomena and rapid variability in lightcurves.
            
        \vspace{0.5em}
        
            \item \textit{\textbf{GROWTH-India}}: Observations and Follow-up campaigns for Gravitational Wave (GW) events from LIGO, Virgo, KAGRA (LVK) collaborations and fast transients using with 0.7m GROWTH-India telescope and collaboration with Zwicky Transient Facility (ZTF) led by Caltech. 
            \item Following up transient X-ray binaries undergoing outbursts.
            \item Daily scanning for fast transients in ZTF data through \texttt{ZTFRest}.
            
        % \vspace{0.5em}
            
        %         \item \textit{\textbf{Daksha Mission}\textit{(Proposed)}}: Studying how the delay in data transfer can affect various science goals.
        %         \item Calculating effect on science goals if only one of the two satellites in the network is operational.
    \end{subsection}
\vspace{-0.1em}
    \begin{subsection}{2. Visiting Student Researcher}{Research Supervisor - Dr. Santanu Mondal}{\textit{Dec 2022 -- Jan 2023}}{\textit{Indian Institute of Astrophysics}}
        \vspace{0.5em}
        \item Conducted energy-dependent time-averaged temporal analysis of a transient black hole X-ray binary GX 339-4 by utilising archival data from \textit{NICER} and \textit{AstroSat} missions.
        \item Studied energy dependence and time evolution of Quasi periodic Oscillations (QPOs) and their harmonic components in the power density spectrum. 
        \item Developed pipelines energy dependent and time resolved temporal studies of persistent surces. 
        \item \textbf{Recipient of IIA Visiting Students Fellowship (2022)}, Indian Institute of Astrophysics, Bangalore.
    \end{subsection}
\end{section}
